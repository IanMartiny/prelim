\section{Conclusion}\label{sec:conclusion}
In this survey we reviewed a number of papers focusing on privacy. We examined
numerous tools that enable privacy while restricting the need for trusted
parties. 

First we introduced the concept of Private Information Retrieval, which enables
querying databases without the server being aware of what was requested. There
are two forms of PIR, information theoretic and computation. IT-PIR requires
users to have trust that the servers running the protocol will not collude with
each other (or with clients). This trust can be garnered by allowing volunteers
to run servers (such as PIR-Tor), which requires the information to be obscured
in some way from the volunteers (possibly with partial databases) though this
scenario increases the damage to service providers in the case of collusion
while limiting the likelihood of damage to clients. C-PIR allows users to have
limited to no trust in service providers by using a centralized model but this
model come with a high computation cost for servers, this is a major drawback to
see real world implementations. It appears that in light of a new design for
C-PIR, the current protocols are the best we can expect. As such future research
making use of C-PIR will need to rely on various mitigations of cost such as
limiting of database or epoch size, which allows clients to make use of these
private protocols while limiting the cost for service providers. However such a
solution will limit the privacy enjoyed compared to solutions without these
mitigations. While IT-PIR is faster than C-PIR, it is still slow and needs
additional protocols to make useable in real world scenarios. Some options here
include batching of responses, a query can return multiple values from the
database, that can be processes to limit the information sent to clients to the
value they requested.

We then discussed contact discovery, and the difficulty posed by how to verify
whether information the client possesses is the same as information possessed by
the service, without forcing the client to give this information to the service.
The recent work in this area leverages both PIR and private set intersection
combining techniques, as mentioned above was necessary, to achieve a more
efficient solution. While the results are promising they are not yet usable in a
real world application.

We discussed presence discovery, which focuses on a similar problem as
above, but with more dynamic information, solutions to this problem often
involve fracturing the database into a long-term and short-term look up model,
allowing for the long-term database to be less private and larger while the
short-term database can be queried more privately and more often.

Our last section on privacy tools was on anonymous communication. Anonymous
communication was achieved by making use of DC-Nets, which allow for the sending
of messages while not having to announce who sent the message. Other work that
scales better (to millions) leverages other techniques, such as mixnets while
still relying on at least one honest server.

After our privacy discussions we moved to private computation. These are
techniques which allow for third parties to perform computations on data either
privately or with proof the computation is correct. Our survey covered the
general techniques used in these cases and discussed how to make formal
arguments about the security and correctness of these protocols. We dove into
the specific task of allowing Danish farmers to trade beet farming contracts
without needing to inform the contractors how much each farmer was bidding or
selling at, which leveraged Shamir secret sharing.

Verifiable computation is achievable through the use of zero knowledge proofs
which allow for users to prove they possess knowledge without having to directly
announce their knowledge. Significant work has been done attempting to provide
general proof systems through reducing general programs to circuits which can be
attested and verified quickly. While these general proof systems are constantly
improving in efficiency and usability, they still lack true generality, often
restricting the proof system to very specific types of problems, such as fixed
size matrix multiplication. More successful work has used zero knowledge proofs
to prove specific knowledge, such as with Zerocoin which solely focuses on
proving knowledge of a user having previously bought some coin (and not spent it
yet).

Differential privacy introduces the notion of privacy within a database.
Questions asked of database are sensitive to the information they contain,
attackers can use this to determine information about individuals in the
database. RAPPOR introduces a protocol of querying databases that allows useful
answers to be given to questions posed to a database while obscuring information
about individuals. This is done by taking the true answer and generating a
permanent random entry that will be used as the basis for a second randomization
round for all repetitions of the query.

Privacy tools allow individuals to communicate or compute values while
maintaining control of their data. Privacy tools that leverage mathematics as
opposed to a trusted party allow users to have complete faith in the system they
use, often at the cost of computation time.