\section{Privacy}\label{sec:priv}
In this section we analyze some of the tools which focus on privacy and
anonymity for average users. Specifically we examine the problems of contact
discovery (how to users of a service discovery whether contacts also use a
service), presence discovery (whether users are currently online), and anonymous
communication.

\subsection{Private Information Retrieval}\label{ssec:PIR}
We briefly introduce the notion of Private Information Retrieval (PIR) as it
crops up in many of the future topics. There are two main classes of PIR,
Information Theoretic (IT-PIR) and Computational (C-PIR). Private information
retrieval was introduced in~\cite{chor1995private} as a means of querying a
database without explicitly informing the server which element was queried.

As is common with privacy schemes there is a simple naive approach: the client
requests the entire database and locally searches for the desired entry. Of
course this method of PIR is infeasible. There has been numerous work in
providing useable and efficient PIR with acceptable threat
models~\cite{aguilar2016xpir,devet2012optimally,beimel2002robust} as well as
providing implementations of uses of PIR, such as PIR-Tor~\cite{mittal2011pir}
which provides an implementation of Tor to gain information on Tor nodes from a
small subset of onion routers using PIR as opposed to using a Peer-to-Peer
network.

\subsubsection{Information Theoretic Private Information Retrieval}
IT-PIR focuses on solving the problem of private database look-ups by
introducing multiple non-colluding servers. These schemes have multiple
(sometimes partial) copies of the database on multiple servers. The client then
constructs specific queries to each server and combines the results.

For example, a client interested in the 3rd entry of a database may make a
request for the \texttt{xor} of the 3rd, 10th and 25th entries of the database
from one server and the \texttt{xor} of the 10th and the 25th entries of the
database from a second server. Locally the client can \texttt{xor} the  replies
from the servers to obtain the 3rd entry. As long as the servers do not discuss
what was requested from each neither knows what entry was actually desired.

Recent work~\cite{demmler2018pir} leverages Distributed Point Functions which
enables querying over the full database in a quick (but private) way.

Previous work has also examined how robust IT-PIR is both in terms of server
failure~\cite{beimel2002robust} and in terms of malicious
servers~\cite{devet2012optimally} to still provide functionality when a
threshold of servers are either purposely non-responsive, or malformed
responses.

\subsubsection{Computational Private Information Retrieval}
\computationalPIR

C-PIR allows for private database queries by leveraging homomorphic encryption. Homomorphic encryption has an encryption function $E(\cdot)$, and decryption function $D(\cdot)$ with the following property:
\[
    D(E(a) + E(b)) = a + b
\]
This allows for querying from a database privately as demonstrated in
Figure~\ref{fig:cpir}. The client can create a vector of encryptions of 0s and a
1 in the location of the entry of the desired file. Upon receiving this
encrypted query the server can use homomorphic multiplication (repeated addition
$D(c\cdot E(a)) = c\cdot a$) and the homomorphic addition property to return the
requested file while learning nothing of \textit{which} file was requested.

Unfortunately C-PIR is computationally expensive, often to the point of
infeasibility for practical use. Though previous
work~\cite{aguilar2016xpir,dong2014fast,gentry2005single} has attempted to
develop and implement efficient C-PIR any current close-to-practical protocol
limits the size of the database queried in C-PIR, or uses C-PIR in conjunction
with other privacy preserving methods (such as Private Set
Intersection~\cite{freedman2004efficient}).

\subsection{Contact discovery}\label{ssec:contactDiscovery}
Contact discovery is the process of clients of a service discovering whether any
of their contacts also use the same service.

In non-private settings this is very simple to achieve, the client provides
contact details to the service which then queries its database and returns who
is available on the service. This method requires each client to reveal their
whole contact list to the service. Allowing the service to track metadata on
every client, as well as leaking non-clients contact information.

Signal systems has gone through the process of trying to handle contact
discovery in a private way~\cite{marlinspike_2014}. As a text messaging service
offering end-to-end encryption it is necessary for Signal to be aware of which
of a users contacts are also Signal clients. They address the difficulties of
contact discovery, a natural first attempt is to simply hash contacts and send
them to the service to verify hashes. Unfortunately the pre-image space of these
hashes is small (there are only 10 billion possible phone numbers, many of which
are not valid, due to area codes) which allows for off-line attacks by the
service after receiving a client's contact list. Other possible options for
contact discovery exist including using Private Information Retrieval (PIR),
bloom filters, and Private Set Intersection (PSI), though these solutions cause
services to make trade-offs due to either large overhead in communication sizes
or in computation costs. Signal has recently implemented a more private contact
discovery method leveraging Intel's Software Guard Extensions
(SGX)~\cite{marlinspike_2017}. The high-level implementation is that clients
make a secure connection to the Signal server and perform a remote attestation
that the code they are running is the published open-source code, and then use
SGX's trusted enclave to perform an intersection where the server is sent an
encrypted form of the clients contacts (which are only decrypted and readable in
the trusted enclave). 

However, recent attacks on SGX~\cite{van2018foreshadow} have demonstrated the
the trusted enclave should not be so trusted. This leads to the natural question
of ``are there better methods of contact discovery?''. 

As stated above, the use of PIR or PSI as a method of private contact discovery
has large overhead which is not usable for large services. However, recent
work~\cite{demmler2018pir} has combined these two concepts to lessen the
overhead. Using a distributed point function, clients can generate an
information theoretic private information retrieval (IT-PIR) request from 2-4
separate (non-colluding) servers, afterward this request is completed by using a
Private Equality Test (PEQ) between the client's contact information and the
database query. This method scales better than previous methods in that a client
with 1024 contacts can perform private contact discovery with a server of 67
million clients in 1.36 seconds and 4.28MiB of communication.

\subsection{Presence discovery}
Presence discovery is the concept a client of a service discovering whether any
of their ``friends'' or contacts on the service are currently online. Again
naively, this can be done by simply requesting information on each contact,
which leaks metadata.

This can be done privately with PIR, but as expected this does not scale well.
The standard threat model for presence discovery is that only your friends
should be able to query about your online status and you should not have to
announce your friends, so there is more work than simply private querying, there
needs to be checks on who can gain presence information.

% talk about DP5 in more detail
DP5~\cite{borisov2015dp5} has implemented a tool using PIR for presence
discovery. By breaking up the size of the database that must be queried PIR
methods can still be effectively used. DP5 works by having a user upload an
identifier and a single symmetric key encrypted under a different public key for
each ``friend'' on their contact list to a long-term (on the order of a day) key
database. Additionally a single entry (containing online status) encrypted under
the common symmetric key is store in a short-term (updated every few minutes)
database. 

The short-term database only has at most one entry per user, which allows for
effective use of IT-PIR. The long-term database contains many entries per user
which scales poorly with PIR, fortunately it only needs to be queried at most
once per day, making the costs not too prohibitive. 

The dual nature of the ``online status'' of clients allows for certain
trade-offs, while the queries can be done privately and (relatively)
efficiently, removing friends can only be done on the order of the long-term
database. As online status is encrypted under a single symmetric key for all of
a clients friends, this status is available to all people that receive an entry
in the long-term database.

\subsection{Anonymous communication}\label{sec:anonymousCommunication}
Free communication is the backbone of freedom. Unfortunately not everyone has
the freedom of speech, these people have the need to speak anonymously to avoid
persecution. Even those with freedom of speech often find the need to speak
anonymously.

There are tools which allow for anonymous communication such as
Tor~\cite{syverson2004tor} which provide anonymity by routing traffic through
relay nodes using onion routing. There are drawbacks such as vulnerability to
traffic analysis and the need to rely on enough volunteers so that a malicious
adversary does not gain majority control of the network. Even with these
drawbacks Tor offers the most practical identity protection. 

\DCNet

Dissent~\cite{wolinsky2012dissent} offers a protocol founded on more secure
communication protocols, at the cost of scalability. Dissent is based on the
security of DC-Nets~\cite{chaum1988dining}, DC-Nets are briefly explained in
Figure~\ref{fig:dcnet}.

While the DC-Nets provide a strong guarantee of anonymity (among honest
participants) they do not scale well. Early implementations of DC-Nets only
scaled to group sizes of about 50~\cite{corrigan2010dissent, goel2003herbivore},
Dissent scales to 5,000 participants using group sizes of 600 participants.

In general, DC-Nets scale poorly due to the need for every client to share
secrets. Dissent solves this by having clients only share secrets with servers.
Communication happens in rounds, each client has some selection of bits it is
allowed to modify every epoch. Upon deciding to send a message in a given epoch
the client \texttt{xor}s its message into their allocated bits for the epoch
along with all shared keys with participating servers. The client then waits for
an attestation of the form $(r, m, sig)$ where $r$ is the current epoch, $m$ is
the collection of clear-text messages from the epoch $r$ and $sig$ is the
attestation to those messages from the servers.

The servers work analogously, \texttt{xor}ing received messages with shared
secrets between all known clients (not only clients that sent messages). They
attest to their result and send it to all other servers, who \texttt{xor} all
the results together which is then attested to and the epoch, messages, and
attestation is sent back to the clients.

Due to the method of communication, a client noticing that their messages are
being censored by a given server can simply communicate with a different
(hopefully) more honest server, allowing Dissent to act under an Anytrust model,
the protocol works as long as one honest server exists. Clients can also accuse
other clients of publishing in their slots, where servers can announce their
random shared secrets corresponding to those bits, to deduce who the offending
client is.

Other work~\cite{van2015vuvuzela} allows for millions of users to communicate
anonymously by making use of mixnets as well as onion routing among a chain of
servers using ``dead drops'' to anonymous communication. Previous
work~\cite{borisov2004off} builds on top of standard encryption schemes to
create Off-The-Record communication which allows for participants (as well as
eavesdroppers) to know who they are communicating with in the moment but not be
able to prove it later.
