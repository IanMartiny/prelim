\section{Privacy}\label{sec:priv}
In this section we analyze some of the tools which focus on privacy and
anonymity for average users. Specifically we examine the problems of contact
discovery (how to users of a service discovery whether contacts also use a
service), presence discovery (whether users are currently online), and anonymous
communication.

\subsection{Private Information Retrieval}\label{ssec:PIR}
We briefly introduce the notion of Private Information Retrieval (PIR) as it
crops up in many of the future topics. There are two main classes of PIR,
Information Theoretic (IT-PIR) and Computational (C-PIR). Private information
retrieval was introduced in~\cite{chor1995private} as a means of querying a
database without explicitly informing the server which element was queried.

As is common with privacy schemes there is a simple naive approach: the client
requests the entire database and locally searches for the desired entry. Of
course this method of PIR is infeasible. There has been numerous work in
providing useable and efficient PIR with acceptable threat models
usable~\cite{aguilar2016xpir,devet2012optimally,beimel2002robust}

\subsubsection{Information Theoretic Private Information Retrieval}
IT-PIR focuses on solving the problem of private database look-ups by
introducing multiple non-colluding servers. These schemes have multiple
(sometimes partial) copies of the database on multiple servers. The client then
constructs specific queries to each server and combines the results.

For example, a client interested in the 3rd entry of a database may make a
request for the \texttt{xor} of the 3rd, 10th and 25th entries of the database
from one server and the \texttt{xor} of the 10th and the 25th entries of the
database from a second server. Locally the client can \texttt{xor} the  replies
from the servers to obtain the 3rd entry. As long as the servers do not discuss
what was requested from each neither knows what entry was actually desired.

Recent work~\cite{demmler2018pir}

% also talk about optimally robust 

\subsubsection{Computation Private Information Retrieval}


\subsection{Contact discovery}\label{ssec:contactDiscovery}
Contact discovery is the process of clients of a service discovering whether any
of their contacts also use the same service.

In non-private settings this is very simple to achieve, the client provides
contact details to the service which then queries its database and returns who
is available on the service. This method requires each client to reveal their
whole contact list to the service. Allowing the service to track metadata on
every client, as well as leaking non-clients phone numbers.

Signal systems has gone through the process of trying to handle contact
discovery in a private way~\cite{marlinspike_2014}. As a text messaging service
offering end-to-end encryption it is necessary for Signal to be aware of which
contact are also Signal clients. They address the difficulties of contact
discovery, a natural first attempt is to simply hash contacts and send them to
the service to verify hashes. Unfortunately the pre-image space of these hashes
is small (there are only 10 billion possible phone numbers, many of which are
not valid, due to area codes) which allows to off-line attacks by the service
after receiving a client's contact list. Other possible options for contact
discovery exist including using Private Information Retrieval (PIR), bloom
filters, and Private Set Intersection (PSI), though these solutions cause
services to make trade-offs due to either large overhead in communication sizes
or in computation costs. Signal has recently implemented a more private contact
discovery method leveraging Intel's Software Guard Extensions
(SGX)~\cite{marlinspike_2017}. The high-level implementation is that clients
make a secure connection to the Signal server and perform a remote attestation
that the code they are running is the published open-source code, and then use
SGX's trusted enclave to perform an intersection where the server is sent an
encrypted form of the clients contacts (which are only decrypted and readable in
the trusted enclave). 

However, recent attacks on SGX~\cite{van2018foreshadow} have demonstrated the
the trusted enclave should not be so trusted. This leads to the natural question
of ``are there better methods of contact discovery?''. 

As stated above, the use of PIR or PSI as a method of private contact discovery
has large overhead which is not usable for large services. However, recent
work~\cite{demmler2018pir} has combined these two concepts to lessen the
overhead. Using a distributed point function, clients can generate an
information theoretic private information retrieval (IT-PIR) request from 2-4
separate servers, afterward this request is completed by using a Private
Equality Test (PEQ) between the client's contact information and the database
query. This method scales better than previous methods in that a client with
1024 contacts can perform private contact discovery with a server of 67 million
clients in 1.36 seconds and 4.28MiB of communication.

% \subsection{Presence discovery}
% Presence discovery is the concept a client of service discovering whether any of
% their contacts also use the service or whether any are currently online
% (depending on the system).